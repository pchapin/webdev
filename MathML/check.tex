%%%%%%%%%%%%%%%%%%%%%%%%%%%%%%%%%%%%%%%%%%%%%%%%%%%%%%%%%%%%%%%%%%%%%%%%%%%%
% FILE         : check.tex
% AUTHOR       : (C) Copyright 2003 by Peter Chapin
% LAST REVISED : 2003-07-28
% SUBJECT      : I use this document to check things out.
%
%
% Send comments or bug reports to:
%
%       Peter Chapin
%       Vermont Technical College
%       Randolph Center, VT 05061
%       pchapin@ecet.vtc.edu
%%%%%%%%%%%%%%%%%%%%%%%%%%%%%%%%%%%%%%%%%%%%%%%%%%%%%%%%%%%%%%%%%%%%%%%%%%%%

%+++++++++++++++++++++++++++++++++
% Preamble and global declarations
%+++++++++++++++++++++++++++++++++
\documentclass{article}

\usepackage{bbold}

\setlength{\parindent}{0em}
\setlength{\parskip}{1.75ex plus0.5ex minus0.5ex}

%++++++++++++++++++++
% The document itself
%++++++++++++++++++++
\begin{document}

%-----------------------
% Title page information
%-----------------------
\title{Check Document}
\author{Peter Chapin}
\date{July 28, 2003}
\maketitle

\section{Introduction}

The \emph{eigenvalues} of a matrix $A \in \textbb{C}^{n \times n}$ are
the $n$ roots of its \emph{characteristic polynomial} $p(z) = \det(zI -
A)$. The set of these roots is called the \emph{spectrum} and is denoted
by $\lambda(A)$. If $\lambda(A) = \{\lambda_1, \ldots, \lambda_n\}$,
then it follows that

\begin{displaymath}
  \det(A) = \lambda_1 \lambda_2 \cdots \lambda_n .
\end{displaymath}

\end{document}
